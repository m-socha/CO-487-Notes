\documentclass[12pt,titlepage]{article}
\usepackage[margin=1in]{geometry}
\usepackage{parskip}

\let\stdsection\section
\renewcommand\section{\clearpage\stdsection}

\usepackage{hyperref}
\hypersetup{
  linktoc=all
}

\begin{document}
  \begin{titlepage}
    \vspace*{\fill}
    \centering

    \textbf{\Huge CO 487 Course Notes} \\ [0.4em]
    \textbf{\Large Applied Cryptography} \\ [1em]
    \textbf{\Large Michael Socha} \\ [1em]
    \textbf{\large University of Waterloo} \\
    \textbf{\large Winter 2019} \\
    \vspace*{\fill}
  \end{titlepage}

  \newpage 

  \pagenumbering{roman}

  \tableofcontents

  \newpage

  \pagenumbering{arabic}

  \section{Course Overview}
    This course is an applied introduction to modern cryptography. Topics covered include:
    \begin{itemize}
      \item Symmetric-key encryption
      \item Hash functions
      \item Authenticated encryption
      \item Public-key encryption
      \item Signature schemes
      \item Key establishment
      \item Key management
      \item Examples of deployed cryptography (e.g. SSL, cryptocurrencies, WPA)
    \end{itemize}

  \section{Introduction - What is Cryptography?}
    Information security (also known as cybersecurity) deals with protecting information assets from
    unauthorized acquisition, damage, disclosure, manipulation, loss, or use. Cryptography deals with
    the mathematical, algorithmic and implementation aspects of information security.

    Cybersecurity more broadly includes the study of computer security, network security and software
    security.

    \subsection{Goals of Cryptography}
      In short, cryptography is about securing communications in the face of malicious adversaries.
      When describing cryptographic scenarios, Alice and Bob are used to indicate two parties who wish
      to communicate with one another across some channel, while Eve is a malicious adversary. Eve may
      attempt to read or modify the data being transmitted.

      The main goals of cryptography are to provide:
      \begin{itemize}
        \item \textbf{Confidentiality:} Keeps data secret from unauthorized entities.
        \item \textbf{Data integrity:} Ensures data has not been altered by unauthorized means.
        \item \textbf{Data origin authentication:} Determines the sender of data.
        \item \textbf{Non-repudiation:} Provides proof of data origin and integrity, which can be
          used to prevent senders from disputing a previous action.
      \end{itemize}

\end{document}
